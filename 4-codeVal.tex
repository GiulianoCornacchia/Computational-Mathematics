
\section{Codice prodotto e validazione}
In questa sezione descriviamo il codice che implementa il metodo PDIP, la generazione delle istanze dei problemi e gli esperimenti.

\subsection{Codice \texttt{MATLAB}}
La nostra soluzione è composta da cinque files:

\begin{itemize}
    \item \texttt{genProblem.m} script che genera un'instanza del problema $(Q,q,A,b)$ usando funzioni ausiliare definite negli altri script.
    \item \texttt{genQF.m} script che dati dimensione $n$, $\delta \in (0,1]$ e un vettore $v\geq0$ genera un vettore $q\in\mathbb{R}^n$ ed una matrice $Q \in \mathbb{R}^{n \times n}$ con densità $\delta$ e autovalori $v$.
    Per generare $Q$ con queste proprietà è stata utilizzata la funzione \texttt{MATLAB} \texttt{sprandsym}.
    
   \item \texttt{generateAdisjointed.m} script che genera la matrice dei vincoli $A$ dato il numero di simplessi $m$ e il numero di variabili $n$; il metodo si assicura che la matrice generata rappresenti un insieme di partizioni di indici disgiunti e che ogni insieme $I^h$ contenga almeno due indici, questo per evitare il soddisfacimento banale di tale vincolo.
    %Per garantire che $rank(A)=m$ e che ogni simplesso contenga almeno due vertici la matrice $A$ viene generata come segue:
    \item \texttt{feasible\_sp.m} script che dati $Q, \;q$ ed $A$ calcola il punto iniziale $(x^0, \lambda_{eq}^0, \lambda_s^0)$ come mostrato in \ref{cap:sp}.
    
    \item \texttt{PDIP.m} script che implementa il metodo primale duale del punto interno come in Algoritmo \ref{alg:pseudo}.
    
\end{itemize}

Di seguito, in Codice 4.1, riportiamo un esempio di utilizzo dei files appena descritti che risolve un problema come quello in analisi, di dimensione $n=1000$ con $k=200$ vincoli e densità di $Q$ pari a $0.4$, usando LDL come metodo per la risoluzione del sistema KKT.
\newpage
\noindent
\texttt{>> n = 1000; m = 200; delta = 0.4;}\\
\texttt{>> p = genProblem(n, m, delta);}\\
\texttt{>> PDIP(p, 100, 1e-14, \virg{ldl});}\\
\noindent
\texttt{Primal-Dual Interior Point method}
\begin{table}[!h]
\begin{tabular}{ccccccllllllllllllll}
\texttt{iter}                                              & \texttt{fval}                                                                  & \texttt{gap}                                                                   & \texttt{dualf}                                                                 & \texttt{primalf}              & \texttt{sTx}                  &  &  &  &  &  &  &  &  &  &  &  &  &  &  \\
\multicolumn{6}{l}{$-----------------------------------------$}                                                                                                                                                                                                                                                                                                         &  &  &  &  &  &  &  &  &  &  &  &  &  &  \\
\texttt{1}                                                 & \texttt{1.669e+01}                                                             & \texttt{2.470e+01}                                                             & \texttt{4.274e+01}                                                             & \texttt{7.364e-16}            & \texttt{4.124e+02}            &  &  &  &  &  &  &  &  &  &  &  &  &  &  \\
\texttt{2}                                                 & \texttt{-6.276e+00}                                                            & \texttt{2.969e+01}                                                             & \texttt{3.032e+00}                                                             & \texttt{7.135e-15}            & \texttt{1.863e+02}            &  &  &  &  &  &  &  &  &  &  &  &  &  &  \\
\texttt{3}                                                 & \texttt{-3.370e+01}                                                            & \texttt{2.461e+00}                                                             & \texttt{9.211e+00}                                                             & \texttt{9.905e-15}            & \texttt{8.294e+01}            &  &  &  &  &  &  &  &  &  &  &  &  &  &  \\
\texttt{\begin{tabular}[c]{@{}c@{}}.\\ .\\ .\end{tabular}} &  \multicolumn{1}{l}{\texttt{}} & \multicolumn{1}{l}{\texttt{}} &  &  &  &  &  &  &  &  &  &  &  &  &  &  \\
\texttt{21}                                                & \texttt{-6.565e+01}                                                            & \texttt{2.554e-13}                                                             & \texttt{1.201e-12}                                                             & \texttt{1.138e-15}            & \texttt{1.677e-11}            &  &  &  &  &  &  &  &  &  &  &  &  &  &  \\
\texttt{22}                                                & \texttt{-6.565e+01}                                                            & \texttt{4.200e-14}                                                             & \texttt{1.979e-13}                                                             & \texttt{1.196e-15}            & \texttt{2.754e-12}            &  &  &  &  &  &  &  &  &  &  &  &  &  &  \\
\texttt{23}                                                & \texttt{-6.565e+01}                                                            & \texttt{6.927e-15}                                                             & \texttt{3.322e-14}                                                             & \texttt{1.053e-15}            & \texttt{4.522e-13}            &  &  &  &  &  &  &  &  &  &  &  &  &  & 
\end{tabular}
\end{table}

\noindent
\texttt{Execution terminated because duality gap reduced under the threshold\\
Primal-Dual Interior Point method terminated in 23 iterations\\
elapsed time is 2.3502 seconds\\
fval = -6.565e+01 and complementary gap = 6.927e-15} 
\begin{center}
    {\small\textbf{Codice 4.1:} esempio di esecuzione del codice prodotto.}
\end{center}
\subsection{Esperimenti}
Allo scopo di testare l'implementazione proposta sono stati effettuati esperimenti considerando PDIP sia con risoluzione iterativa (\emph{PDIP-GMRES}) che diretta (\textit{PDIP-LDL}) del sistema lineare.
Inoltre tempi ed accuratezza del metodo da noi implementato sono stati confrontati con quegli della funzione \textit{built-in} di \texttt{MATLAB} \texttt{quadprog}.

I parametri coinvolti nella nostra analisi sono:
\begin{itemize}
    \item dimensione dell'input $n$, ovvero la dimensione della matrice $Q$;
    \item $m$, percentuale rispetto ad $n$ di vincoli; abbiamo anche che $k=m\cdot n = rank(A) = |I| \leq n/2$;
    \item densità $\delta$ della matrice $Q$: la percentuale di elementi non-nulli in $Q$.
\end{itemize}

Gli esperimenti possono essere divisi in quattro gruppi:
\begin{enumerate}
    \item nel primo insieme di esperimenti abbiamo voluto testare la scalabilità del nostro metodo fissando $m$ e densità al variare di $n$
    \item nel secondo gruppo di esperimenti abbiamo voluto investigare l'effetto del numero dei vincoli sul tempo di convergenza; abbiamo fatto variare $m$ lasciando fisse $n$ e densità.
    \item nel terzo gruppo di esperimenti abbiamo investigato gli effetti della densità della matrice $Q$ fissati $m$ ed $n$
    \item infine il quarto tipo di esperimenti è mirato a valutare sia la convergenza dell'implementazione proposta, che l'accuratezza della soluzione trovata; quindi, fissati $n, \; m$ e $\delta$, analizziamo l'andamento del gap e la differnza fra le soluzioni del nostro metodo e \texttt{quadprog}
\end{enumerate}

La Tabella \ref{tab:param} riassume i valori assegnati ai parametri durante gli esperimenti.
\begin{table}[!h]
\centering
\begin{tabular}{c|c|c|c}
\textbf{Esperimento}                & \textbf{$\mathbf{n}$}                                         & \textbf{$\mathbf{m}$}                 & \textbf{$\mathbf{\delta}$}     \\ \hline
\textbf{Scalabilità}                &  \begin{tabular}[c]{@{}c@{}}$[50, 75, 100, 200, 300, \dots\; , \\ 1000, 1250, 1500, 1750, 2000]$\end{tabular}                    & $40\%$ & $40\%$                        \\ \hline
\textbf{N° Vincoli}                 & $1000$                                                        & $[5\%, 10\%, 20\%, 30\%, 40\%, 50\%]$ & $40\%$                         \\ \hline
\textbf{Densità}                    & $1000$                                                        & $20\%$                                & $[10\%, 20\%, \dots\;, 100\%]$ \\ \hline
\textbf{\begin{tabular}[c]{@{}c@{}}Convergenza \&\\  Accuratezza\end{tabular}} & $1000$                                                        & $20\%$                                & $40\%$                        
\end{tabular}
\caption{Tabella che riassume i parametri coinvolti negli esperimenti.\label{tab:param}}
\end{table}

Per ogni esperimento sono state effettuate 20 ripetizioni per poi calcolare media e deviazione standard delle metriche.
Tali esperimenti sono stati eseguiti su un Laptop con processore \texttt{Intel(R) Core(TM) i5-3337U CPU @ 1.80GHz (4 CPUs)} e 6 GB di RAM. 