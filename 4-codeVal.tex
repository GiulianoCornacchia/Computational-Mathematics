
\section{Codice prodotto e validazione}
In questa sezione descriviamo il codice che implementa il metodo PDIP, la generazione delle istanze dei problemi e gli esperimenti.

\subsection{Codice \texttt{MATLAB}}
La nostra soluzione è composta da cinque files:

\begin{itemize}
    \item \texttt{genProblem.m} script che genera un'instanza del problema $(Q,q,A,b)$ usando funzioni ausiliare definite negli altri script.
    \item \texttt{genQF.m} script che dati dimensione $n$, $\delta \in (0,1]$ e un vettore $v\geq0$ genera un vettore $q\in\mathbb{R}^n$ ed una matrice $Q \in \mathbb{R}^{n \times n}$ con densità $\delta$ e autovalori $v$.
    Per generare $Q$ con queste proprietà è stata utilizzata la funzione \texttt{MATLAB} \texttt{sprandsym}.
    
   \item \texttt{generateAdisjointed.m} script che genera la matrice dei vincoli $A$ dato il numero di simplessi $m$ e il numero di variabili $n$.
    %Per garantire che $rank(A)=m$ e che ogni simplesso contenga almeno due vertici la matrice $A$ viene generata come segue:
    \item \texttt{feasible\_sp.m} script che dati $Q, \;q$ ed $A$ calcola il punto iniziale $(x^0, \lambda_{eq}^0, \lambda_s^0)$ come mostrato in \ref{cap:sp}.
    
    \item \texttt{PDIP.m} script che implementa il metodo primale duale del punto interno come in Algoritmo \ref{alg:pseudo}.
    
\end{itemize}


\subsection{Esperimenti}
Allo scopo di testare l'implementazione proposta sono stati effettuati esperimenti considerando PDIP sia con risoluzione iterativa che diretta del sistema lineare.
Inoltre tempi ed accuratezza del metodo da noi implementato sono stati confrontati con la funzione \textit{built-in} di \texttt{MATLAB} \texttt{quadprog}.
Gli esperimenti possono essere divisi in tre gruppi:
\begin{itemize}
    \item Nel primo insieme di esperimenti abbiamo voluto testare la scalabilità del nostro metodo fissando $m$ e densità al variare di $n$.
    \item Nel secondo gruppo di esperimenti abbiamo voluto investigare l'effetto del numero dei vincoli sulla convergenza.
    Abbiamo fatto variare $m$ lasciando fisse $n$ e densità.
    \item infine, nel terzo gruppo di esperimenti abbiamo investigato gli effetti della densità della matrice $Q$ fissati $m$ ed $n$.
\end{itemize}

Per ogni esperimento sono state effettuate 20 ripetizioni per poi calcolare media e deviazione standard delle metriche.
Tali esperimenti sono stati eseguiti su un Laptop con processore \textit{Intel(R) Core(TM) i5-3337U CPU @ 1.80GHz (4 CPUs)} e 6 GB di RAM. 